% Arquivo usado como padrão para todos os arquivos LaTex
% Criado em 17/12/22 às 18h00

% Preâmbulo necessário
\usepackage[portuguese]{babel}
%\usepackage[brazilian]{babel}
\usepackage[utf8]{inputenc}
\usepackage[T1]{fontenc}
\usepackage{amsmath}
\usepackage{amsfonts}
\usepackage{amssymb}
\usepackage{makeidx}
\usepackage{xcolor}
\usepackage{mdframed} % Pacote que dá cor e bordas aos textos
\usepackage{pdfpages} % Pacote para auxiliar incluir pdf externo ao documento
\usepackage{titlesec} % Pacote que auxilia a troca de caracteristicas do section

\usepackage{enumerate} % Pacote que permite alterar os itens do enumerate
\usepackage{pgfplots} % Pacote de plotar gráficos
\usepackage{tikz} % Pacote para desenho
\usepackage{booktabs} % Pacote para usar as tabelas exportadas do excel
\usepackage{multirow} % Pacote para mesclar linhas em tabelas
\usepackage{wrapfig} % Pacote para inserir figuras e texto lado a lado
\usepackage{multicol} % Pacote para inserir colunas
%\usepackage[x11names]{xcolor} % Pacote para inserir sombreamento
% \setlength{\columnsep}{1cm} % Dá uma maior separação nos textos em colunas

\everymath{\displaystyle} % Faz com que os símbolos matemáticos fiquem "melhores" ..tipo insere o limite com x \to abaixo do lim

\graphicspath{{img/}} % Fixa a pasta img para o caminho das imagens quando está na raiz do projeto.


\usepackage{graphicx,color} % Pacote necessário para inserção de figuras e textos coloridos

\usepackage{indentfirst} % Pacote que insere espaço de parágrafo
\usepackage{setspace} % Pacote necessário para espaçamento entre linhas \singlespacing (simples) \onehalfspacing (1.5) \doublespacing (2.0)
\linespread{1.5}% Espaçamento entre as linhas de 1.5 
\usepackage[left=2cm,right=2cm,top=2cm,bottom=3cm]{geometry}
%\usepackage{caladea} % Um tipo de fonte
\usepackage{arev} % Fonte para todo corpo de texto
\usepackage{arevmath} % Mesma fonte do corpo de texto para as fórmulas
\usepackage[normalem]{ulem} % Serve para hifenizar melhor
%\pagestyle{headings} % Serve para colocar numeração na parte superior
\usepackage{lipsum} % Usado para inserir texto falso com o comando \lipsum[num-num]
\usepackage{cancel} % Faz o 'cancelado' \ em equações


% Abre um ambiente Quadro igual os de tabela
\usepackage{trivfloat}
% \trivfloat{quadro} % Pode usar assim e fazer com que a palavra 'quadro' seja o nome do ambiente

% Quando compilado para pdf esse pacote fará o indice ficar com link
\usepackage[colorlinks=true, pdfstartview=FitV, linkcolor=black,citecolor=black, urlcolor=black]{hyperref}

% Esse comando altera o formato e cor das \section 
\titleformat{\section}{\color{blue}\large\bfseries}{\color{red}\thesection}{0.5em}{}
%\titleformat{\subsection}{\color{black}\Large\bfseries}{\color{red}\thesection}{0.5em}{}

\usepackage{fancyhdr} % Altera formato de cabeçalho, rodapé e capítulos

% A opção é o tipo de formatação que queremos usar:
% Sonny;	% Lenny;	% Glenn;	% Rejne;	% Bjarne;	% Bjornstrup

%\usepackage[Conny]{fncychap}
\usepackage[Sonny]{fncychap}
%\usepackage[Glenn]{fncychap}
%\usepackage[Rejne]{fncychap}
%\usepackage[Bjarne]{fncychap}
%\usepackage[Bjornstrup]{fncychap}
%\usepackage[Lenny]{fncychap}


% **** Trabalhar com códigos de programação em Python ****
\usepackage{listingsutf8}
\usepackage{listings} % Para trabalhar com códigos de programação.


%\lstset{language=Python} % Escolhendo a linguagem



% Novos comandos para otimizar a escrita
\newcommand{\rr}{\mathbb{R}^2}
\newcommand{\rrr}{\mathbb{R}^3}
\newcommand{\R}{\mathbb{R}}

% Vetor combinação linear : inserir apenas uma letra \vcl{p}
\newcommand{\vcl}[1]{\overrightarrow{#1} = \overrightarrow{#1_1} + \overrightarrow{#1_2} + \cdots + \overrightarrow{#1_n} }

\newcommand{\matriz}[4]{\begin{bmatrix} #1 & #2\\ #3 & #4 \end{bmatrix}}
\newcommand{\vet}[1]{\overrightarrow{#1}}

% Casos especiais de vetores u, v e w
\newcommand{\vetu}{\overrightarrow{u}}
\newcommand{\vetv}{\overrightarrow{v}}
\newcommand{\vetw}{\overrightarrow{w}}

% Quando usar o arquivo teo.tex para o pacote mdframed 
\newtheorem{deff}{Definição}[section]
\newtheorem{exe}{Exemplo}[section]
\newtheorem{teorema}{Teorema}[section]
%\newtheorem{exer}{Exercício}[section]


