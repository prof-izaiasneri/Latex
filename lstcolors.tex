
% ********************************************************
% **** Trabalhar com códigos de programação em Python ****
% ********************************************************

% Para definição das cores acesse o site => http://latexcolor.com/
% Definindo as cores
\definecolor{mygreen}{rgb}{0,0.6,0} 
\definecolor{mygray}{rgb}{0.5,0.5,0.5}
\definecolor{mymauve}{rgb}{0.58,0,0.82}

\definecolor{codegreen}{rgb}{0,0.6,0}
\definecolor{codegray}{rgb}{0.5,0.5,0.5}
\definecolor{codepurple}{rgb}{0.58,0,0.82}
\definecolor{backcolour}{rgb}{0.95,0.95,0.92}


% Para usar em códigos Python
\lstdefinestyle{mypython}{ 
		backgroundcolor=\color{cyan!10}, % choose the background color
		basicstyle=\large, % size of fonts used for the code
		breaklines=true,   % automatic line breaking only at whitespace
		captionpos=b,      % sets the caption-position to bottom
		commentstyle=\color{mygreen}, % comment style
		escapeinside={\%*}{*)},       % if you want to add LaTeX within your code
		keywordstyle=\color{blue},    % keyword style
		stringstyle=\color{mymauve},  % string literal style
		tabsize = 1,				  % Altera a posição inicial do código
		% emphstyle=\color{red},
		showstringspaces=false,		  % Tira o 'risco' dos espaços.
		inputencoding=utf8/latin1,
		literate={ó}{{\'o}}1 {Ó}{{\'O}}1 {ã}{{\~a}}1 {ç}{{\c c}}1 {á}{{\' a}}1 {é}{{\' e}}1 {ú}{{\' u}}1 {ê}{{\^ e}}1, 
}
% **** Trabalhar com códigos de programação em Python ****


% Para usar em códigos C
\lstdefinestyle{myc}{
	inputencoding=utf8/latin1,
	backgroundcolor=\color{backcolour},   
	commentstyle=\color{codegreen},
	keywordstyle=\color{magenta},
	numberstyle=\tiny\color{codegray},
	stringstyle=\color{codepurple},
	basicstyle=\ttfamily\footnotesize,
	breakatwhitespace=false,         
	breaklines=true,                 
	captionpos=b,                    
	keepspaces=true,                 
	numbers=left,                    
	numbersep=5pt,                  
	showspaces=false,                
	showstringspaces=false,
	showtabs=false,                  
	tabsize=2,
	literate={ó}{{\'o}}1 {Ó}{{\'O}}1 {ã}{{\~a}}1 {ç}{{\c c}}1 {á}{{\' a}}1 {é}{{\' e}}1 {ú}{{\' u}}1 {ê}{{\^ e}}1, 
}
